\documentclass[12pt, a4paper]{article}
\usepackage[russian]{babel}
\usepackage{fontspec}
\setmainfont[
  Ligatures=TeX,
  Extension=.otf,
  BoldFont=cmunbx,
  ItalicFont=cmunti,
  BoldItalicFont=cmunbi,
]{cmunrm}
\usepackage{polyglossia}
\setdefaultlanguage{russian}
\setotherlanguage{english}


\usepackage{geometry}

\geometry{
margin=1.5cm
}

\usepackage{indentfirst}

\usepackage{arydshln}
\usepackage[fleqn]{amsmath}
\usepackage{xfrac}
\usepackage{esint}
\usepackage{amssymb}
\usepackage{mathbbol}
\usepackage[T1]{fontenc}
\usepackage{mathtools}
\usepackage{color}
\usepackage{ulem}
\usepackage{tabu}
\usepackage{multirow}
\usepackage{rotating}

\usepackage[outline]{contour}
\contourlength{1.2pt}

\usepackage{tikz}
\usepackage{graphics}
\usepackage{xcolor}

\usepackage[at]{easylist}

\DeclareMathOperator{\sign}{sign}

\begin{document}

\section{3 лаба}

\subsection{Задание}

МС - местное сопротивление

ТП - трубопровод

$x_1$, $x_2$ - координаты при которых происходит переключение канала.

Ударник сам переключает свои каналы.

\subsection{Метод работы перегородки}

$ x(t) $

$ x^k, x^{k+1}, x^{k+2}, \dots $

$ x^k < x_\text{окр.} \wedge x^{k+1} > x_\text{окр.} $


\noindent\begin{easylist}
\ListProperties(Hang1=true, Start1=1)
@ Уменьшаем $\sfrac{ht}{2}$.
@ Мы могли попасть справа или слева $x_\text{окр.}$
@ Уменшаем пока $x$ не окажется слева.
@ Остановиться, когда подходим на расстояние $\Delta\delta = 10^{-6} \text{м}$, тогда $x = x_{\text{окр.}}$, $v = 0$.
\end{easylist}

\subsection{Константы:}

\noindent\begin{easylist}
\ListProperties(Hang1=true, Start1=1)
@ $ \nu = 10^{-6} $
@ $ \gamma = 1.4 $
@ $ E_s = \rho (\rho_{swd})^2 \approx 1260000 $ - модуль жесткости воды.
@ $ C = \frac{V}{E_s} $
@ $ C_{cav} = 10^{\frac5\gamma}\frac{V}{\gamma} $
@ $ d \approx 2 \text{см} $
@ $ R_e^{crt} = 321 $
@ $ \text{\ae} = 1.75 $
@ $ R_e^{mult} = \frac{dH}{\nu S} $
@ $ dH = \frac{4S}{\pi d} $
@ $ B = \frac{1}{l\sqrt{2g}} $
@ $ F = S\sqrt{2g} $
@ $ \xi = \text{WTF?} $
@ $ r = \frac{12g\nu l}{(dH)^2 S} $
@ $ r_\text{\ae} = \frac{0.1582 g \nu^{0.25} l}{(dH)^{1.35} S^{1.75}} $
\end{easylist}

\subsection{Формулы:}

\noindent\begin{easylist}
\ListProperties(Hang1=true, Start1=1)
@ $ p' = h \Phi(q_1 - q_j,\ p,\ C,\ C_{cav}) $
@ $ q' = h G(p_i-p_j-P_\alpha(q),\ q) $
@ $
\Phi(q_{ij},\ p,\ c,\ C) = \left\{
\begin{aligned}
&\frac{p^{\left(1+\frac{1}{\gamma}\right)}q_{ij}}{C_{cav}},  & 0 < p \leqslant p_{\text{Атм.}},\ 0 < q_{ij}, \\
& \frac{q_{ij}}{C}, & \text{иначе.}
\end{aligned}
\right.
$
@ $
P_\alpha(q) = \left\{
\begin{aligned}
&rq,  &\text{WTF?}, \\
&r_\text{\ae}|q|^\text{\ae}\sign(q), & \text{иначе.}
\end{aligned}
\right.
$
@ $ G(dp,\ q) = B\sqrt{|dp|}(F\sqrt{\frac{dp}{\xi}}^3 - q) $
\end{easylist}

\subsection{Обозначение:}

\noindent\begin{easylist}
\ListProperties(Hang1=true, Start1=1)
@ $ V $ - объем камеры. 
@ $ S $ - площадь сечения местного сопротивления.
@ $ C $ - жесткость воды.
@ $ C_{cav} $ - жесткость воды при кавитации.
\end{easylist}

\end{document}